
% Default to the notebook output style

    


% Inherit from the specified cell style.




    
\documentclass[11pt]{article}

    
    
    \usepackage[T1]{fontenc}
    % Nicer default font (+ math font) than Computer Modern for most use cases
    \usepackage{mathpazo}

    % Basic figure setup, for now with no caption control since it's done
    % automatically by Pandoc (which extracts ![](path) syntax from Markdown).
    \usepackage{graphicx}
    % We will generate all images so they have a width \maxwidth. This means
    % that they will get their normal width if they fit onto the page, but
    % are scaled down if they would overflow the margins.
    \makeatletter
    \def\maxwidth{\ifdim\Gin@nat@width>\linewidth\linewidth
    \else\Gin@nat@width\fi}
    \makeatother
    \let\Oldincludegraphics\includegraphics
    % Set max figure width to be 80% of text width, for now hardcoded.
    \renewcommand{\includegraphics}[1]{\Oldincludegraphics[width=.8\maxwidth]{#1}}
    % Ensure that by default, figures have no caption (until we provide a
    % proper Figure object with a Caption API and a way to capture that
    % in the conversion process - todo).
    \usepackage{caption}
    \DeclareCaptionLabelFormat{nolabel}{}
    \captionsetup{labelformat=nolabel}

    \usepackage{adjustbox} % Used to constrain images to a maximum size 
    \usepackage{xcolor} % Allow colors to be defined
    \usepackage{enumerate} % Needed for markdown enumerations to work
    \usepackage{geometry} % Used to adjust the document margins
    \usepackage{amsmath} % Equations
    \usepackage{amssymb} % Equations
    \usepackage{textcomp} % defines textquotesingle
    % Hack from http://tex.stackexchange.com/a/47451/13684:
    \AtBeginDocument{%
        \def\PYZsq{\textquotesingle}% Upright quotes in Pygmentized code
    }
    \usepackage{upquote} % Upright quotes for verbatim code
    \usepackage{eurosym} % defines \euro
    \usepackage[mathletters]{ucs} % Extended unicode (utf-8) support
    \usepackage[utf8x]{inputenc} % Allow utf-8 characters in the tex document
    \usepackage{fancyvrb} % verbatim replacement that allows latex
    \usepackage{grffile} % extends the file name processing of package graphics 
                         % to support a larger range 
    % The hyperref package gives us a pdf with properly built
    % internal navigation ('pdf bookmarks' for the table of contents,
    % internal cross-reference links, web links for URLs, etc.)
    \usepackage{hyperref}
    \usepackage{longtable} % longtable support required by pandoc >1.10
    \usepackage{booktabs}  % table support for pandoc > 1.12.2
    \usepackage[inline]{enumitem} % IRkernel/repr support (it uses the enumerate* environment)
    \usepackage[normalem]{ulem} % ulem is needed to support strikethroughs (\sout)
                                % normalem makes italics be italics, not underlines
    

    
    
    % Colors for the hyperref package
    \definecolor{urlcolor}{rgb}{0,.145,.698}
    \definecolor{linkcolor}{rgb}{.71,0.21,0.01}
    \definecolor{citecolor}{rgb}{.12,.54,.11}

    % ANSI colors
    \definecolor{ansi-black}{HTML}{3E424D}
    \definecolor{ansi-black-intense}{HTML}{282C36}
    \definecolor{ansi-red}{HTML}{E75C58}
    \definecolor{ansi-red-intense}{HTML}{B22B31}
    \definecolor{ansi-green}{HTML}{00A250}
    \definecolor{ansi-green-intense}{HTML}{007427}
    \definecolor{ansi-yellow}{HTML}{DDB62B}
    \definecolor{ansi-yellow-intense}{HTML}{B27D12}
    \definecolor{ansi-blue}{HTML}{208FFB}
    \definecolor{ansi-blue-intense}{HTML}{0065CA}
    \definecolor{ansi-magenta}{HTML}{D160C4}
    \definecolor{ansi-magenta-intense}{HTML}{A03196}
    \definecolor{ansi-cyan}{HTML}{60C6C8}
    \definecolor{ansi-cyan-intense}{HTML}{258F8F}
    \definecolor{ansi-white}{HTML}{C5C1B4}
    \definecolor{ansi-white-intense}{HTML}{A1A6B2}

    % commands and environments needed by pandoc snippets
    % extracted from the output of `pandoc -s`
    \providecommand{\tightlist}{%
      \setlength{\itemsep}{0pt}\setlength{\parskip}{0pt}}
    \DefineVerbatimEnvironment{Highlighting}{Verbatim}{commandchars=\\\{\}}
    % Add ',fontsize=\small' for more characters per line
    \newenvironment{Shaded}{}{}
    \newcommand{\KeywordTok}[1]{\textcolor[rgb]{0.00,0.44,0.13}{\textbf{{#1}}}}
    \newcommand{\DataTypeTok}[1]{\textcolor[rgb]{0.56,0.13,0.00}{{#1}}}
    \newcommand{\DecValTok}[1]{\textcolor[rgb]{0.25,0.63,0.44}{{#1}}}
    \newcommand{\BaseNTok}[1]{\textcolor[rgb]{0.25,0.63,0.44}{{#1}}}
    \newcommand{\FloatTok}[1]{\textcolor[rgb]{0.25,0.63,0.44}{{#1}}}
    \newcommand{\CharTok}[1]{\textcolor[rgb]{0.25,0.44,0.63}{{#1}}}
    \newcommand{\StringTok}[1]{\textcolor[rgb]{0.25,0.44,0.63}{{#1}}}
    \newcommand{\CommentTok}[1]{\textcolor[rgb]{0.38,0.63,0.69}{\textit{{#1}}}}
    \newcommand{\OtherTok}[1]{\textcolor[rgb]{0.00,0.44,0.13}{{#1}}}
    \newcommand{\AlertTok}[1]{\textcolor[rgb]{1.00,0.00,0.00}{\textbf{{#1}}}}
    \newcommand{\FunctionTok}[1]{\textcolor[rgb]{0.02,0.16,0.49}{{#1}}}
    \newcommand{\RegionMarkerTok}[1]{{#1}}
    \newcommand{\ErrorTok}[1]{\textcolor[rgb]{1.00,0.00,0.00}{\textbf{{#1}}}}
    \newcommand{\NormalTok}[1]{{#1}}
    
    % Additional commands for more recent versions of Pandoc
    \newcommand{\ConstantTok}[1]{\textcolor[rgb]{0.53,0.00,0.00}{{#1}}}
    \newcommand{\SpecialCharTok}[1]{\textcolor[rgb]{0.25,0.44,0.63}{{#1}}}
    \newcommand{\VerbatimStringTok}[1]{\textcolor[rgb]{0.25,0.44,0.63}{{#1}}}
    \newcommand{\SpecialStringTok}[1]{\textcolor[rgb]{0.73,0.40,0.53}{{#1}}}
    \newcommand{\ImportTok}[1]{{#1}}
    \newcommand{\DocumentationTok}[1]{\textcolor[rgb]{0.73,0.13,0.13}{\textit{{#1}}}}
    \newcommand{\AnnotationTok}[1]{\textcolor[rgb]{0.38,0.63,0.69}{\textbf{\textit{{#1}}}}}
    \newcommand{\CommentVarTok}[1]{\textcolor[rgb]{0.38,0.63,0.69}{\textbf{\textit{{#1}}}}}
    \newcommand{\VariableTok}[1]{\textcolor[rgb]{0.10,0.09,0.49}{{#1}}}
    \newcommand{\ControlFlowTok}[1]{\textcolor[rgb]{0.00,0.44,0.13}{\textbf{{#1}}}}
    \newcommand{\OperatorTok}[1]{\textcolor[rgb]{0.40,0.40,0.40}{{#1}}}
    \newcommand{\BuiltInTok}[1]{{#1}}
    \newcommand{\ExtensionTok}[1]{{#1}}
    \newcommand{\PreprocessorTok}[1]{\textcolor[rgb]{0.74,0.48,0.00}{{#1}}}
    \newcommand{\AttributeTok}[1]{\textcolor[rgb]{0.49,0.56,0.16}{{#1}}}
    \newcommand{\InformationTok}[1]{\textcolor[rgb]{0.38,0.63,0.69}{\textbf{\textit{{#1}}}}}
    \newcommand{\WarningTok}[1]{\textcolor[rgb]{0.38,0.63,0.69}{\textbf{\textit{{#1}}}}}
    
    
    % Define a nice break command that doesn't care if a line doesn't already
    % exist.
    \def\br{\hspace*{\fill} \\* }
    % Math Jax compatability definitions
    \def\gt{>}
    \def\lt{<}
    % Document parameters
    \title{Chapter 2 Homework 02}
    
    
    

    % Pygments definitions
    
\makeatletter
\def\PY@reset{\let\PY@it=\relax \let\PY@bf=\relax%
    \let\PY@ul=\relax \let\PY@tc=\relax%
    \let\PY@bc=\relax \let\PY@ff=\relax}
\def\PY@tok#1{\csname PY@tok@#1\endcsname}
\def\PY@toks#1+{\ifx\relax#1\empty\else%
    \PY@tok{#1}\expandafter\PY@toks\fi}
\def\PY@do#1{\PY@bc{\PY@tc{\PY@ul{%
    \PY@it{\PY@bf{\PY@ff{#1}}}}}}}
\def\PY#1#2{\PY@reset\PY@toks#1+\relax+\PY@do{#2}}

\expandafter\def\csname PY@tok@w\endcsname{\def\PY@tc##1{\textcolor[rgb]{0.73,0.73,0.73}{##1}}}
\expandafter\def\csname PY@tok@c\endcsname{\let\PY@it=\textit\def\PY@tc##1{\textcolor[rgb]{0.25,0.50,0.50}{##1}}}
\expandafter\def\csname PY@tok@cp\endcsname{\def\PY@tc##1{\textcolor[rgb]{0.74,0.48,0.00}{##1}}}
\expandafter\def\csname PY@tok@k\endcsname{\let\PY@bf=\textbf\def\PY@tc##1{\textcolor[rgb]{0.00,0.50,0.00}{##1}}}
\expandafter\def\csname PY@tok@kp\endcsname{\def\PY@tc##1{\textcolor[rgb]{0.00,0.50,0.00}{##1}}}
\expandafter\def\csname PY@tok@kt\endcsname{\def\PY@tc##1{\textcolor[rgb]{0.69,0.00,0.25}{##1}}}
\expandafter\def\csname PY@tok@o\endcsname{\def\PY@tc##1{\textcolor[rgb]{0.40,0.40,0.40}{##1}}}
\expandafter\def\csname PY@tok@ow\endcsname{\let\PY@bf=\textbf\def\PY@tc##1{\textcolor[rgb]{0.67,0.13,1.00}{##1}}}
\expandafter\def\csname PY@tok@nb\endcsname{\def\PY@tc##1{\textcolor[rgb]{0.00,0.50,0.00}{##1}}}
\expandafter\def\csname PY@tok@nf\endcsname{\def\PY@tc##1{\textcolor[rgb]{0.00,0.00,1.00}{##1}}}
\expandafter\def\csname PY@tok@nc\endcsname{\let\PY@bf=\textbf\def\PY@tc##1{\textcolor[rgb]{0.00,0.00,1.00}{##1}}}
\expandafter\def\csname PY@tok@nn\endcsname{\let\PY@bf=\textbf\def\PY@tc##1{\textcolor[rgb]{0.00,0.00,1.00}{##1}}}
\expandafter\def\csname PY@tok@ne\endcsname{\let\PY@bf=\textbf\def\PY@tc##1{\textcolor[rgb]{0.82,0.25,0.23}{##1}}}
\expandafter\def\csname PY@tok@nv\endcsname{\def\PY@tc##1{\textcolor[rgb]{0.10,0.09,0.49}{##1}}}
\expandafter\def\csname PY@tok@no\endcsname{\def\PY@tc##1{\textcolor[rgb]{0.53,0.00,0.00}{##1}}}
\expandafter\def\csname PY@tok@nl\endcsname{\def\PY@tc##1{\textcolor[rgb]{0.63,0.63,0.00}{##1}}}
\expandafter\def\csname PY@tok@ni\endcsname{\let\PY@bf=\textbf\def\PY@tc##1{\textcolor[rgb]{0.60,0.60,0.60}{##1}}}
\expandafter\def\csname PY@tok@na\endcsname{\def\PY@tc##1{\textcolor[rgb]{0.49,0.56,0.16}{##1}}}
\expandafter\def\csname PY@tok@nt\endcsname{\let\PY@bf=\textbf\def\PY@tc##1{\textcolor[rgb]{0.00,0.50,0.00}{##1}}}
\expandafter\def\csname PY@tok@nd\endcsname{\def\PY@tc##1{\textcolor[rgb]{0.67,0.13,1.00}{##1}}}
\expandafter\def\csname PY@tok@s\endcsname{\def\PY@tc##1{\textcolor[rgb]{0.73,0.13,0.13}{##1}}}
\expandafter\def\csname PY@tok@sd\endcsname{\let\PY@it=\textit\def\PY@tc##1{\textcolor[rgb]{0.73,0.13,0.13}{##1}}}
\expandafter\def\csname PY@tok@si\endcsname{\let\PY@bf=\textbf\def\PY@tc##1{\textcolor[rgb]{0.73,0.40,0.53}{##1}}}
\expandafter\def\csname PY@tok@se\endcsname{\let\PY@bf=\textbf\def\PY@tc##1{\textcolor[rgb]{0.73,0.40,0.13}{##1}}}
\expandafter\def\csname PY@tok@sr\endcsname{\def\PY@tc##1{\textcolor[rgb]{0.73,0.40,0.53}{##1}}}
\expandafter\def\csname PY@tok@ss\endcsname{\def\PY@tc##1{\textcolor[rgb]{0.10,0.09,0.49}{##1}}}
\expandafter\def\csname PY@tok@sx\endcsname{\def\PY@tc##1{\textcolor[rgb]{0.00,0.50,0.00}{##1}}}
\expandafter\def\csname PY@tok@m\endcsname{\def\PY@tc##1{\textcolor[rgb]{0.40,0.40,0.40}{##1}}}
\expandafter\def\csname PY@tok@gh\endcsname{\let\PY@bf=\textbf\def\PY@tc##1{\textcolor[rgb]{0.00,0.00,0.50}{##1}}}
\expandafter\def\csname PY@tok@gu\endcsname{\let\PY@bf=\textbf\def\PY@tc##1{\textcolor[rgb]{0.50,0.00,0.50}{##1}}}
\expandafter\def\csname PY@tok@gd\endcsname{\def\PY@tc##1{\textcolor[rgb]{0.63,0.00,0.00}{##1}}}
\expandafter\def\csname PY@tok@gi\endcsname{\def\PY@tc##1{\textcolor[rgb]{0.00,0.63,0.00}{##1}}}
\expandafter\def\csname PY@tok@gr\endcsname{\def\PY@tc##1{\textcolor[rgb]{1.00,0.00,0.00}{##1}}}
\expandafter\def\csname PY@tok@ge\endcsname{\let\PY@it=\textit}
\expandafter\def\csname PY@tok@gs\endcsname{\let\PY@bf=\textbf}
\expandafter\def\csname PY@tok@gp\endcsname{\let\PY@bf=\textbf\def\PY@tc##1{\textcolor[rgb]{0.00,0.00,0.50}{##1}}}
\expandafter\def\csname PY@tok@go\endcsname{\def\PY@tc##1{\textcolor[rgb]{0.53,0.53,0.53}{##1}}}
\expandafter\def\csname PY@tok@gt\endcsname{\def\PY@tc##1{\textcolor[rgb]{0.00,0.27,0.87}{##1}}}
\expandafter\def\csname PY@tok@err\endcsname{\def\PY@bc##1{\setlength{\fboxsep}{0pt}\fcolorbox[rgb]{1.00,0.00,0.00}{1,1,1}{\strut ##1}}}
\expandafter\def\csname PY@tok@kc\endcsname{\let\PY@bf=\textbf\def\PY@tc##1{\textcolor[rgb]{0.00,0.50,0.00}{##1}}}
\expandafter\def\csname PY@tok@kd\endcsname{\let\PY@bf=\textbf\def\PY@tc##1{\textcolor[rgb]{0.00,0.50,0.00}{##1}}}
\expandafter\def\csname PY@tok@kn\endcsname{\let\PY@bf=\textbf\def\PY@tc##1{\textcolor[rgb]{0.00,0.50,0.00}{##1}}}
\expandafter\def\csname PY@tok@kr\endcsname{\let\PY@bf=\textbf\def\PY@tc##1{\textcolor[rgb]{0.00,0.50,0.00}{##1}}}
\expandafter\def\csname PY@tok@bp\endcsname{\def\PY@tc##1{\textcolor[rgb]{0.00,0.50,0.00}{##1}}}
\expandafter\def\csname PY@tok@fm\endcsname{\def\PY@tc##1{\textcolor[rgb]{0.00,0.00,1.00}{##1}}}
\expandafter\def\csname PY@tok@vc\endcsname{\def\PY@tc##1{\textcolor[rgb]{0.10,0.09,0.49}{##1}}}
\expandafter\def\csname PY@tok@vg\endcsname{\def\PY@tc##1{\textcolor[rgb]{0.10,0.09,0.49}{##1}}}
\expandafter\def\csname PY@tok@vi\endcsname{\def\PY@tc##1{\textcolor[rgb]{0.10,0.09,0.49}{##1}}}
\expandafter\def\csname PY@tok@vm\endcsname{\def\PY@tc##1{\textcolor[rgb]{0.10,0.09,0.49}{##1}}}
\expandafter\def\csname PY@tok@sa\endcsname{\def\PY@tc##1{\textcolor[rgb]{0.73,0.13,0.13}{##1}}}
\expandafter\def\csname PY@tok@sb\endcsname{\def\PY@tc##1{\textcolor[rgb]{0.73,0.13,0.13}{##1}}}
\expandafter\def\csname PY@tok@sc\endcsname{\def\PY@tc##1{\textcolor[rgb]{0.73,0.13,0.13}{##1}}}
\expandafter\def\csname PY@tok@dl\endcsname{\def\PY@tc##1{\textcolor[rgb]{0.73,0.13,0.13}{##1}}}
\expandafter\def\csname PY@tok@s2\endcsname{\def\PY@tc##1{\textcolor[rgb]{0.73,0.13,0.13}{##1}}}
\expandafter\def\csname PY@tok@sh\endcsname{\def\PY@tc##1{\textcolor[rgb]{0.73,0.13,0.13}{##1}}}
\expandafter\def\csname PY@tok@s1\endcsname{\def\PY@tc##1{\textcolor[rgb]{0.73,0.13,0.13}{##1}}}
\expandafter\def\csname PY@tok@mb\endcsname{\def\PY@tc##1{\textcolor[rgb]{0.40,0.40,0.40}{##1}}}
\expandafter\def\csname PY@tok@mf\endcsname{\def\PY@tc##1{\textcolor[rgb]{0.40,0.40,0.40}{##1}}}
\expandafter\def\csname PY@tok@mh\endcsname{\def\PY@tc##1{\textcolor[rgb]{0.40,0.40,0.40}{##1}}}
\expandafter\def\csname PY@tok@mi\endcsname{\def\PY@tc##1{\textcolor[rgb]{0.40,0.40,0.40}{##1}}}
\expandafter\def\csname PY@tok@il\endcsname{\def\PY@tc##1{\textcolor[rgb]{0.40,0.40,0.40}{##1}}}
\expandafter\def\csname PY@tok@mo\endcsname{\def\PY@tc##1{\textcolor[rgb]{0.40,0.40,0.40}{##1}}}
\expandafter\def\csname PY@tok@ch\endcsname{\let\PY@it=\textit\def\PY@tc##1{\textcolor[rgb]{0.25,0.50,0.50}{##1}}}
\expandafter\def\csname PY@tok@cm\endcsname{\let\PY@it=\textit\def\PY@tc##1{\textcolor[rgb]{0.25,0.50,0.50}{##1}}}
\expandafter\def\csname PY@tok@cpf\endcsname{\let\PY@it=\textit\def\PY@tc##1{\textcolor[rgb]{0.25,0.50,0.50}{##1}}}
\expandafter\def\csname PY@tok@c1\endcsname{\let\PY@it=\textit\def\PY@tc##1{\textcolor[rgb]{0.25,0.50,0.50}{##1}}}
\expandafter\def\csname PY@tok@cs\endcsname{\let\PY@it=\textit\def\PY@tc##1{\textcolor[rgb]{0.25,0.50,0.50}{##1}}}

\def\PYZbs{\char`\\}
\def\PYZus{\char`\_}
\def\PYZob{\char`\{}
\def\PYZcb{\char`\}}
\def\PYZca{\char`\^}
\def\PYZam{\char`\&}
\def\PYZlt{\char`\<}
\def\PYZgt{\char`\>}
\def\PYZsh{\char`\#}
\def\PYZpc{\char`\%}
\def\PYZdl{\char`\$}
\def\PYZhy{\char`\-}
\def\PYZsq{\char`\'}
\def\PYZdq{\char`\"}
\def\PYZti{\char`\~}
% for compatibility with earlier versions
\def\PYZat{@}
\def\PYZlb{[}
\def\PYZrb{]}
\makeatother


    % Exact colors from NB
    \definecolor{incolor}{rgb}{0.0, 0.0, 0.5}
    \definecolor{outcolor}{rgb}{0.545, 0.0, 0.0}



    
    % Prevent overflowing lines due to hard-to-break entities
    \sloppy 
    % Setup hyperref package
    \hypersetup{
      breaklinks=true,  % so long urls are correctly broken across lines
      colorlinks=true,
      urlcolor=urlcolor,
      linkcolor=linkcolor,
      citecolor=citecolor,
      }
    % Slightly bigger margins than the latex defaults
    
    \geometry{verbose,tmargin=1in,bmargin=1in,lmargin=1in,rmargin=1in}
    
    

    \begin{document}
    
    
    \maketitle
    
    

    
    \section{Question 01}\label{question-01}

    \begin{Verbatim}[commandchars=\\\{\}]
{\color{incolor}In [{\color{incolor}1}]:} \PY{k+kn}{import} \PY{n+nn}{numpy} \PY{k}{as} \PY{n+nn}{np}
        \PY{k+kn}{import} \PY{n+nn}{matplotlib}\PY{n+nn}{.}\PY{n+nn}{pyplot} \PY{k}{as} \PY{n+nn}{plt} 
        \PY{k+kn}{import} \PY{n+nn}{math}
\end{Verbatim}


    \begin{Verbatim}[commandchars=\\\{\}]
{\color{incolor}In [{\color{incolor}2}]:} \PY{n}{np}\PY{o}{.}\PY{n}{linspace}\PY{p}{(}\PY{l+m+mi}{0}\PY{p}{,} \PY{n}{np}\PY{o}{.}\PY{n}{pi}\PY{p}{,} \PY{n}{num} \PY{o}{=} \PY{l+m+mi}{30}\PY{p}{)}
\end{Verbatim}


\begin{Verbatim}[commandchars=\\\{\}]
{\color{outcolor}Out[{\color{outcolor}2}]:} array([0.        , 0.10833078, 0.21666156, 0.32499234, 0.43332312,
               0.54165391, 0.64998469, 0.75831547, 0.86664625, 0.97497703,
               1.08330781, 1.19163859, 1.29996937, 1.40830016, 1.51663094,
               1.62496172, 1.7332925 , 1.84162328, 1.94995406, 2.05828484,
               2.16661562, 2.2749464 , 2.38327719, 2.49160797, 2.59993875,
               2.70826953, 2.81660031, 2.92493109, 3.03326187, 3.14159265])
\end{Verbatim}
            
    \begin{Verbatim}[commandchars=\\\{\}]
{\color{incolor}In [{\color{incolor}3}]:} \PY{n}{x} \PY{o}{=} \PY{p}{(}\PY{n}{np}\PY{o}{.}\PY{n}{linspace}\PY{p}{(}\PY{l+m+mi}{0}\PY{p}{,} \PY{n}{np}\PY{o}{.}\PY{n}{pi}\PY{p}{,} \PY{n}{num} \PY{o}{=} \PY{l+m+mi}{30}\PY{p}{)}\PY{p}{)}
\end{Verbatim}


    \begin{Verbatim}[commandchars=\\\{\}]
{\color{incolor}In [{\color{incolor}4}]:} \PY{c+c1}{\PYZsh{}\PYZsh{} Define the functions for the overall formulas}
        
        \PY{k}{def} \PY{n+nf}{f}\PY{p}{(}\PY{n}{y}\PY{p}{)}\PY{p}{:}
            \PY{k}{return} \PY{p}{(}\PY{n}{y}\PY{o}{\PYZhy{}}\PY{l+m+mf}{0.9}\PY{o}{*}\PY{n}{np}\PY{o}{.}\PY{n}{sin}\PY{p}{(}\PY{n}{y}\PY{p}{)}\PY{p}{)}
        
        \PY{k}{def} \PY{n+nf}{fp}\PY{p}{(}\PY{n}{y}\PY{p}{)}\PY{p}{:}
            \PY{k}{return} \PY{p}{(}\PY{l+m+mi}{1}\PY{o}{\PYZhy{}}\PY{p}{(}\PY{l+m+mf}{0.9}\PY{o}{*}\PY{n}{np}\PY{o}{.}\PY{n}{cos}\PY{p}{(}\PY{n}{y}\PY{p}{)}\PY{p}{)}\PY{p}{)}
        
        \PY{c+c1}{\PYZsh{}\PYZsh{} Define the function to solve for X. }
        \PY{c+c1}{\PYZsh{}\PYZsh{} Each input needs its own loop, find out which input is being entered and play with iterations}
        \PY{c+c1}{\PYZsh{}\PYZsh{} No need to add a spot in the function for tolerance. It is already set in class as 10**(\PYZhy{}6)}
        
        \PY{k}{def} \PY{n+nf}{Prob1}\PY{p}{(}\PY{n}{n}\PY{p}{,} \PY{n}{Iterations}\PY{p}{)}\PY{p}{:}
            \PY{n}{y} \PY{o}{=} \PY{l+m+mi}{0}
            \PY{n}{i} \PY{o}{=} \PY{l+m+mi}{0}
            \PY{k}{while} \PY{n}{i} \PY{o}{\PYZlt{}}\PY{o}{=} \PY{n}{Iterations}\PY{p}{:}
                \PY{n}{x} \PY{o}{=} \PY{n}{y} \PY{o}{\PYZhy{}} \PY{p}{(}\PY{p}{(}\PY{n}{f}\PY{p}{(}\PY{n}{y}\PY{p}{)}\PY{o}{\PYZhy{}}\PY{n}{n}\PY{p}{)}\PY{o}{/}\PY{n}{fp}\PY{p}{(}\PY{n}{y}\PY{p}{)}\PY{p}{)}
                \PY{k}{if} \PY{n}{np}\PY{o}{.}\PY{n}{absolute}\PY{p}{(}\PY{n}{x} \PY{o}{\PYZhy{}} \PY{n}{y}\PY{p}{)} \PY{o}{\PYZlt{}} \PY{l+m+mi}{10}\PY{o}{*}\PY{o}{*}\PY{p}{(}\PY{o}{\PYZhy{}}\PY{l+m+mi}{6}\PY{p}{)}\PY{p}{:}
                    \PY{n+nb}{print}\PY{p}{(}\PY{n}{x}\PY{p}{)}
                    \PY{k}{break}
                \PY{n}{i} \PY{o}{=} \PY{n}{i} \PY{o}{+} \PY{l+m+mi}{1}
                \PY{n}{y} \PY{o}{=} \PY{n}{x}
            \PY{k}{if} \PY{n}{i} \PY{o}{\PYZgt{}} \PY{n}{Iterations}\PY{p}{:}
                \PY{n+nb}{print}\PY{p}{(}\PY{n}{f}\PY{l+s+s1}{\PYZsq{}}\PY{l+s+s1}{The method failed after N\PYZus{}0 iterations, N\PYZus{}0 = }\PY{l+s+si}{\PYZob{}Iterations\PYZcb{}}\PY{l+s+s1}{\PYZsq{}}\PY{p}{)}
\end{Verbatim}


    \begin{Verbatim}[commandchars=\\\{\}]
{\color{incolor}In [{\color{incolor}5}]:} \PY{c+c1}{\PYZsh{}\PYZsh{} Run the function to solve for x in each iteration}
        \PY{c+c1}{\PYZsh{}\PYZsh{} Self note, had to raise the number of iterations to something high}
        
        \PY{n}{x} \PY{o}{=} \PY{n}{np}\PY{o}{.}\PY{n}{linspace}\PY{p}{(}\PY{l+m+mi}{0}\PY{p}{,} \PY{n}{np}\PY{o}{.}\PY{n}{pi}\PY{p}{,} \PY{n}{num} \PY{o}{=} \PY{l+m+mi}{30}\PY{p}{)}
        \PY{k}{for} \PY{n}{n} \PY{o+ow}{in} \PY{n}{x}\PY{p}{:}
            \PY{n}{Prob1}\PY{p}{(}\PY{n}{n}\PY{p}{,} \PY{l+m+mi}{100}\PY{p}{)}
\end{Verbatim}


    \begin{Verbatim}[commandchars=\\\{\}]
0.0
0.6604701856042979
0.9473407987171497
1.1444131303639302
1.300691844877185
1.433140220934481
1.5497860499774534
1.655117812587781
1.7519234076803416
1.8420654702561892
1.9268572778959292
2.007264078579107
2.0840193067347905
2.1576957200296207
2.2287510506335493
2.297558467587193
2.3644275836436326
2.4296193565068216
2.4933569231885917
2.5558336524672014
2.617219250484121
2.677664476989677
2.7373048536363815
2.7962636311392703
2.854654205967585
2.912582125684288
2.970146786655176
3.027442903313447
3.0845618111153983
3.141592653589793

    \end{Verbatim}

    \section{Question 2}\label{question-2}

    Step 1: Find the max time and amount by getting the derivate and solving
it for zero. Plus the max time into the regular equal to solve for the
max amount.

    \begin{figure}
\centering
\includegraphics{attachment:image.png}
\caption{image.png}
\end{figure}

    Graphed the equation to get an idea of when the equation will be around
.25 amount. Looks like it happens around 11.

    \begin{Verbatim}[commandchars=\\\{\}]
{\color{incolor}In [{\color{incolor}6}]:} \PY{k}{def} \PY{n+nf}{f}\PY{p}{(}\PY{n}{x}\PY{p}{)}\PY{p}{:} 
            \PY{k}{return} \PY{p}{(}\PY{n}{math}\PY{o}{.}\PY{n}{exp}\PY{p}{(}\PY{l+m+mi}{1}\PY{p}{)}\PY{o}{/}\PY{l+m+mi}{3}\PY{p}{)} \PY{o}{*} \PY{n}{x} \PY{o}{*} \PY{p}{(}\PY{n}{math}\PY{o}{.}\PY{n}{exp}\PY{p}{(}\PY{o}{\PYZhy{}}\PY{n}{x}\PY{o}{/}\PY{l+m+mi}{3}\PY{p}{)}\PY{p}{)}
        
        \PY{n}{f2} \PY{o}{=} \PY{n}{np}\PY{o}{.}\PY{n}{vectorize}\PY{p}{(}\PY{n}{f}\PY{p}{)} 
        
        \PY{n}{x} \PY{o}{=} \PY{n}{np}\PY{o}{.}\PY{n}{arange}\PY{p}{(}\PY{l+m+mi}{1}\PY{p}{,} \PY{l+m+mf}{15.1}\PY{p}{,} \PY{l+m+mf}{0.1}\PY{p}{)} 
        
        \PY{n}{plt}\PY{o}{.}\PY{n}{plot}\PY{p}{(}\PY{n}{x}\PY{p}{,} \PY{n}{f2}\PY{p}{(}\PY{n}{x}\PY{p}{)}\PY{p}{)} 
        
        \PY{n}{plt}\PY{o}{.}\PY{n}{show}\PY{p}{(}\PY{p}{)}
\end{Verbatim}


    \begin{center}
    \adjustimage{max size={0.9\linewidth}{0.9\paperheight}}{output_10_0.png}
    \end{center}
    { \hspace*{\fill} \\}
    
    Subtracted the max amount from the base equation. The .25 amount should
now be at the 'x' axis. Plugged it into the Newton method and solved.

    \begin{Verbatim}[commandchars=\\\{\}]
{\color{incolor}In [{\color{incolor}7}]:} \PY{k}{def} \PY{n+nf}{f}\PY{p}{(}\PY{n}{t}\PY{p}{)}\PY{p}{:}
            \PY{k}{return} \PY{p}{(}\PY{n}{math}\PY{o}{.}\PY{n}{exp}\PY{p}{(}\PY{l+m+mi}{1}\PY{p}{)}\PY{o}{/}\PY{l+m+mi}{3}\PY{p}{)} \PY{o}{*} \PY{n}{t} \PY{o}{*} \PY{p}{(}\PY{n}{math}\PY{o}{.}\PY{n}{exp}\PY{p}{(}\PY{o}{\PYZhy{}}\PY{n}{t}\PY{o}{/}\PY{l+m+mi}{3}\PY{p}{)}\PY{p}{)} \PY{o}{\PYZhy{}} \PY{l+m+mf}{0.25}
        
        \PY{k}{def} \PY{n+nf}{fp}\PY{p}{(}\PY{n}{t}\PY{p}{)}\PY{p}{:}
            \PY{k}{return} \PY{o}{\PYZhy{}} \PY{p}{(}\PY{l+m+mi}{1}\PY{o}{/}\PY{l+m+mi}{9}\PY{p}{)} \PY{o}{*} \PY{n}{math}\PY{o}{.}\PY{n}{exp}\PY{p}{(}\PY{l+m+mi}{1}\PY{o}{\PYZhy{}}\PY{p}{(}\PY{n}{t}\PY{o}{/}\PY{l+m+mi}{3}\PY{p}{)}\PY{p}{)} \PY{o}{*} \PY{p}{(}\PY{n}{t}\PY{o}{\PYZhy{}}\PY{l+m+mi}{3}\PY{p}{)}
        
        
        \PY{n}{p\PYZus{}0} \PY{o}{=} \PY{l+m+mi}{11}
        \PY{n}{TOL} \PY{o}{=} \PY{l+m+mi}{1} \PY{o}{*} \PY{l+m+mi}{10}\PY{o}{*}\PY{o}{*}\PY{o}{\PYZhy{}}\PY{l+m+mi}{6}
        \PY{n}{Iterations} \PY{o}{=} \PY{l+m+mi}{20}
        
        \PY{n}{results} \PY{o}{=} \PY{p}{[}\PY{p}{]}   \PY{c+c1}{\PYZsh{}\PYZsh{} Need to save the results to something}
        \PY{n}{results}\PY{o}{.}\PY{n}{append}\PY{p}{(}\PY{n}{p\PYZus{}0}\PY{p}{)}
        \PY{c+c1}{\PYZsh{}\PYZsh{} Step 1 \PYZhy{} Set i = 1}
        
        \PY{n}{i} \PY{o}{=} \PY{l+m+mi}{1}
        
        \PY{c+c1}{\PYZsh{}\PYZsh{} Step 2 \PYZhy{} While (i \PYZlt{}= N\PYZus{}0) do steps 3\PYZhy{}6}
        
        \PY{k}{while} \PY{n}{i} \PY{o}{\PYZlt{}}\PY{o}{=} \PY{n}{Iterations}\PY{p}{:}
        
        \PY{c+c1}{\PYZsh{}\PYZsh{} Step 3 Set p = p\PYZus{}0 \PYZhy{} f(p\PYZus{}0)/f\PYZsq{}(p\PYZus{}0) (Compute p\PYZus{}i)}
        
            \PY{n}{p} \PY{o}{=} \PY{n}{p\PYZus{}0} \PY{o}{\PYZhy{}} \PY{p}{(}\PY{n}{f}\PY{p}{(}\PY{n}{p\PYZus{}0}\PY{p}{)}\PY{o}{/}\PY{n}{fp}\PY{p}{(}\PY{n}{p\PYZus{}0}\PY{p}{)}\PY{p}{)}
        
        \PY{c+c1}{\PYZsh{}\PYZsh{} Step 4 if |p \PYZhy{} p\PYZus{}0| \PYZlt{} TOL then OUTPUT(p); (The procedure was successful.) STOP}
        
            \PY{k}{if} \PY{n}{np}\PY{o}{.}\PY{n}{absolute}\PY{p}{(}\PY{n}{p} \PY{o}{\PYZhy{}} \PY{n}{p\PYZus{}0}\PY{p}{)} \PY{o}{\PYZlt{}} \PY{n}{TOL}\PY{p}{:}
                \PY{n+nb}{print}\PY{p}{(}\PY{n}{f}\PY{l+s+s1}{\PYZsq{}}\PY{l+s+s1}{The procedure was successful. p\PYZus{}0: }\PY{l+s+si}{\PYZob{}p\PYZus{}0\PYZcb{}}\PY{l+s+s1}{ n: }\PY{l+s+si}{\PYZob{}i\PYZcb{}}\PY{l+s+s1}{\PYZsq{}}\PY{p}{)}
                \PY{k}{break}
        \PY{c+c1}{\PYZsh{}\PYZsh{} Step 5 Set i= i + 1.}
            \PY{n}{i} \PY{o}{=} \PY{n}{i} \PY{o}{+} \PY{l+m+mi}{1}
        
        \PY{c+c1}{\PYZsh{}\PYZsh{} Step 6 Set p\PYZus{}0 = p. (Update p\PYZus{}0)}
            \PY{n}{p\PYZus{}0} \PY{o}{=} \PY{n}{p}
            \PY{n}{results}\PY{o}{.}\PY{n}{append}\PY{p}{(}\PY{n}{p}\PY{p}{)}
        
        \PY{c+c1}{\PYZsh{}\PYZsh{} Step 7 OUTPUT (\PYZsq{}The method failed after N\PYZhy{}0 iterations, N\PYZhy{}0 =\PYZsq{}, N\PYZus{}0) }
        \PY{c+c1}{\PYZsh{}\PYZsh{} (The procedure was unsuccessful. STOP)}
        \PY{k}{if} \PY{n}{i} \PY{o}{\PYZgt{}} \PY{n}{Iterations}\PY{p}{:}
            \PY{n+nb}{print}\PY{p}{(}\PY{n}{f}\PY{l+s+s1}{\PYZsq{}}\PY{l+s+s1}{The method failed after N\PYZus{}0 iterations, N\PYZus{}0 = }\PY{l+s+si}{\PYZob{}Iterations\PYZcb{}}\PY{l+s+s1}{\PYZsq{}}\PY{p}{)}
\end{Verbatim}


    \begin{Verbatim}[commandchars=\\\{\}]
The procedure was successful. p\_0: 11.07790354509074 n: 3

    \end{Verbatim}

    The second injection should be given at 11.0078 hours or

    \begin{Verbatim}[commandchars=\\\{\}]
{\color{incolor}In [{\color{incolor}8}]:} \PY{n}{p\PYZus{}0} \PY{o}{*} \PY{l+m+mi}{60}
\end{Verbatim}


\begin{Verbatim}[commandchars=\\\{\}]
{\color{outcolor}Out[{\color{outcolor}8}]:} 664.6742127054443
\end{Verbatim}
            
    \^{} Number of minutes after the first injection.

    Part 3: 75\% of the original injection. Still subtracting .25 from the
formula:

    \begin{Verbatim}[commandchars=\\\{\}]
{\color{incolor}In [{\color{incolor}9}]:} \PY{k}{def} \PY{n+nf}{f}\PY{p}{(}\PY{n}{t}\PY{p}{)}\PY{p}{:}
            \PY{k}{return} \PY{p}{(}\PY{o}{.}\PY{l+m+mi}{75}\PY{p}{)}\PY{o}{*}\PY{p}{(}\PY{n}{math}\PY{o}{.}\PY{n}{exp}\PY{p}{(}\PY{l+m+mi}{1}\PY{p}{)}\PY{o}{/}\PY{l+m+mi}{3}\PY{p}{)} \PY{o}{*} \PY{n}{t} \PY{o}{*} \PY{p}{(}\PY{n}{math}\PY{o}{.}\PY{n}{exp}\PY{p}{(}\PY{o}{\PYZhy{}}\PY{n}{t}\PY{o}{/}\PY{l+m+mi}{3}\PY{p}{)}\PY{p}{)} \PY{o}{\PYZhy{}} \PY{p}{(}\PY{o}{.}\PY{l+m+mi}{25}\PY{p}{)}
        
        \PY{k}{def} \PY{n+nf}{fp}\PY{p}{(}\PY{n}{t}\PY{p}{)}\PY{p}{:}
            \PY{k}{return} \PY{o}{\PYZhy{}} \PY{p}{(}\PY{l+m+mi}{1}\PY{o}{/}\PY{l+m+mi}{9}\PY{p}{)} \PY{o}{*} \PY{n}{math}\PY{o}{.}\PY{n}{exp}\PY{p}{(}\PY{l+m+mi}{1}\PY{o}{\PYZhy{}}\PY{p}{(}\PY{n}{t}\PY{o}{/}\PY{l+m+mi}{3}\PY{p}{)}\PY{p}{)} \PY{o}{*} \PY{p}{(}\PY{n}{t}\PY{o}{\PYZhy{}}\PY{l+m+mi}{3}\PY{p}{)}
        
        
        \PY{n}{p\PYZus{}0} \PY{o}{=} \PY{l+m+mi}{11}
        \PY{n}{TOL} \PY{o}{=} \PY{l+m+mi}{1} \PY{o}{*} \PY{l+m+mi}{10}\PY{o}{*}\PY{o}{*}\PY{o}{\PYZhy{}}\PY{l+m+mi}{6}
        \PY{n}{Iterations} \PY{o}{=} \PY{l+m+mi}{20}
        
        \PY{n}{results} \PY{o}{=} \PY{p}{[}\PY{p}{]}   \PY{c+c1}{\PYZsh{}\PYZsh{} Need to save the results to something}
        \PY{n}{results}\PY{o}{.}\PY{n}{append}\PY{p}{(}\PY{n}{p\PYZus{}0}\PY{p}{)}
        \PY{c+c1}{\PYZsh{}\PYZsh{} Step 1 \PYZhy{} Set i = 1}
        
        \PY{n}{i} \PY{o}{=} \PY{l+m+mi}{1}
        
        \PY{c+c1}{\PYZsh{}\PYZsh{} Step 2 \PYZhy{} While (i \PYZlt{}= N\PYZus{}0) do steps 3\PYZhy{}6}
        
        \PY{k}{while} \PY{n}{i} \PY{o}{\PYZlt{}}\PY{o}{=} \PY{n}{Iterations}\PY{p}{:}
        
        \PY{c+c1}{\PYZsh{}\PYZsh{} Step 3 Set p = p\PYZus{}0 \PYZhy{} f(p\PYZus{}0)/f\PYZsq{}(p\PYZus{}0) (Compute p\PYZus{}i)}
        
            \PY{n}{p} \PY{o}{=} \PY{n}{p\PYZus{}0} \PY{o}{\PYZhy{}} \PY{p}{(}\PY{n}{f}\PY{p}{(}\PY{n}{p\PYZus{}0}\PY{p}{)}\PY{o}{/}\PY{n}{fp}\PY{p}{(}\PY{n}{p\PYZus{}0}\PY{p}{)}\PY{p}{)}
        
        \PY{c+c1}{\PYZsh{}\PYZsh{} Step 4 if |p \PYZhy{} p\PYZus{}0| \PYZlt{} TOL then OUTPUT(p); (The procedure was successful.) STOP}
        
            \PY{k}{if} \PY{n}{np}\PY{o}{.}\PY{n}{absolute}\PY{p}{(}\PY{n}{p} \PY{o}{\PYZhy{}} \PY{n}{p\PYZus{}0}\PY{p}{)} \PY{o}{\PYZlt{}} \PY{n}{TOL}\PY{p}{:}
                \PY{n+nb}{print}\PY{p}{(}\PY{n}{f}\PY{l+s+s1}{\PYZsq{}}\PY{l+s+s1}{The procedure was successful. p\PYZus{}0: }\PY{l+s+si}{\PYZob{}p\PYZus{}0\PYZcb{}}\PY{l+s+s1}{ n: }\PY{l+s+si}{\PYZob{}i\PYZcb{}}\PY{l+s+s1}{\PYZsq{}}\PY{p}{)}
                \PY{k}{break}
        \PY{c+c1}{\PYZsh{}\PYZsh{} Step 5 Set i= i + 1.}
            \PY{n}{i} \PY{o}{=} \PY{n}{i} \PY{o}{+} \PY{l+m+mi}{1}
        
        \PY{c+c1}{\PYZsh{}\PYZsh{} Step 6 Set p\PYZus{}0 = p. (Update p\PYZus{}0)}
            \PY{n}{p\PYZus{}0} \PY{o}{=} \PY{n}{p}
            \PY{n}{results}\PY{o}{.}\PY{n}{append}\PY{p}{(}\PY{n}{p}\PY{p}{)}
        
        \PY{c+c1}{\PYZsh{}\PYZsh{} Step 7 OUTPUT (\PYZsq{}The method failed after N\PYZhy{}0 iterations, N\PYZhy{}0 =\PYZsq{}, N\PYZus{}0) }
        \PY{c+c1}{\PYZsh{}\PYZsh{} (The procedure was unsuccessful. STOP)}
        \PY{k}{if} \PY{n}{i} \PY{o}{\PYZgt{}} \PY{n}{Iterations}\PY{p}{:}
            \PY{n+nb}{print}\PY{p}{(}\PY{n}{f}\PY{l+s+s1}{\PYZsq{}}\PY{l+s+s1}{The method failed after N\PYZus{}0 iterations, N\PYZus{}0 = }\PY{l+s+si}{\PYZob{}Iterations\PYZcb{}}\PY{l+s+s1}{\PYZsq{}}\PY{p}{)}
\end{Verbatim}


    \begin{Verbatim}[commandchars=\\\{\}]
The procedure was successful. p\_0: 9.867844878232496 n: 11

    \end{Verbatim}

    \begin{Verbatim}[commandchars=\\\{\}]
{\color{incolor}In [{\color{incolor}10}]:} \PY{l+m+mf}{11.0779035450907} \PY{o}{+} \PY{l+m+mf}{9.86784487823250}
\end{Verbatim}


\begin{Verbatim}[commandchars=\\\{\}]
{\color{outcolor}Out[{\color{outcolor}10}]:} 20.945748423323202
\end{Verbatim}
            
    \^{} hr

    \section{Question 3}\label{question-3}

    \subsection{Newton's Method}\label{newtons-method}

    \begin{Verbatim}[commandchars=\\\{\}]
{\color{incolor}In [{\color{incolor}11}]:} \PY{c+c1}{\PYZsh{}\PYZsh{} Newton\PYZsq{}s method}
         \PY{c+c1}{\PYZsh{}\PYZsh{} Testing for pg.68 }
         \PY{c+c1}{\PYZsh{}\PYZsh{} Chapter 2.3 Example 1}
         
         \PY{k}{def} \PY{n+nf}{f}\PY{p}{(}\PY{n}{x}\PY{p}{)}\PY{p}{:}
             \PY{k}{return} \PY{n}{x}\PY{o}{*}\PY{o}{*}\PY{l+m+mi}{2} \PY{o}{\PYZhy{}} \PY{l+m+mi}{2}
         
         \PY{k}{def} \PY{n+nf}{fp}\PY{p}{(}\PY{n}{x}\PY{p}{)}\PY{p}{:}
             \PY{k}{return} \PY{l+m+mi}{2}\PY{o}{*}\PY{n}{x}
         
         \PY{c+c1}{\PYZsh{}\PYZsh{} INPUT initial approximation p\PYZus{}0; tolerance TOL; maximum number of iterations N\PYZus{}0.}
         
         \PY{n}{p\PYZus{}0} \PY{o}{=} \PY{l+m+mi}{1}
         \PY{n}{TOL} \PY{o}{=} \PY{l+m+mi}{1} \PY{o}{*} \PY{l+m+mi}{10}\PY{o}{*}\PY{o}{*}\PY{o}{\PYZhy{}}\PY{l+m+mi}{6}
         \PY{n}{Iterations} \PY{o}{=} \PY{l+m+mi}{10}
         
         \PY{n}{results} \PY{o}{=} \PY{p}{[}\PY{p}{]}   \PY{c+c1}{\PYZsh{}\PYZsh{} Need to save the results to something}
         \PY{c+c1}{\PYZsh{}\PYZsh{} Step 1 \PYZhy{} Set i = 1}
         
         \PY{n}{i} \PY{o}{=} \PY{l+m+mi}{1}
         
         \PY{c+c1}{\PYZsh{}\PYZsh{} Step 2 \PYZhy{} While (i \PYZlt{}= N\PYZus{}0) do steps 3\PYZhy{}6}
         
         \PY{k}{while} \PY{n}{i} \PY{o}{\PYZlt{}}\PY{o}{=} \PY{n}{Iterations}\PY{p}{:}
         
         \PY{c+c1}{\PYZsh{}\PYZsh{} Step 3 Set p = p\PYZus{}0 \PYZhy{} f(p\PYZus{}0)/f\PYZsq{}(p\PYZus{}0) (Compute p\PYZus{}i)}
         
             \PY{n}{p} \PY{o}{=} \PY{n}{p\PYZus{}0} \PY{o}{\PYZhy{}} \PY{p}{(}\PY{n}{f}\PY{p}{(}\PY{n}{p\PYZus{}0}\PY{p}{)}\PY{o}{/}\PY{n}{fp}\PY{p}{(}\PY{n}{p\PYZus{}0}\PY{p}{)}\PY{p}{)}
         
         \PY{c+c1}{\PYZsh{}\PYZsh{} Step 4 if |p \PYZhy{} p\PYZus{}0| \PYZlt{} TOL then OUTPUT(p); (The procedure was successful.) STOP}
         
             \PY{k}{if} \PY{n}{np}\PY{o}{.}\PY{n}{absolute}\PY{p}{(}\PY{n}{p} \PY{o}{\PYZhy{}} \PY{n}{p\PYZus{}0}\PY{p}{)} \PY{o}{\PYZlt{}} \PY{n}{TOL}\PY{p}{:}
                 \PY{n+nb}{print}\PY{p}{(}\PY{n}{f}\PY{l+s+s1}{\PYZsq{}}\PY{l+s+s1}{The procedure was successful. p\PYZus{}0: }\PY{l+s+si}{\PYZob{}p\PYZus{}0\PYZcb{}}\PY{l+s+s1}{ n: }\PY{l+s+s1}{\PYZob{}}\PY{l+s+s1}{i\PYZhy{}1\PYZcb{}}\PY{l+s+s1}{\PYZsq{}}\PY{p}{)}
                 \PY{k}{break}
         \PY{c+c1}{\PYZsh{}\PYZsh{} Step 5 Set i= i + 1.}
             \PY{n}{i} \PY{o}{=} \PY{n}{i} \PY{o}{+} \PY{l+m+mi}{1}
         
         \PY{c+c1}{\PYZsh{}\PYZsh{} Step 6 Set p\PYZus{}0 = p. (Update p\PYZus{}0)}
             \PY{n}{p\PYZus{}0} \PY{o}{=} \PY{n}{p}
             \PY{n}{results}\PY{o}{.}\PY{n}{append}\PY{p}{(}\PY{n}{p}\PY{p}{)}
         
         \PY{c+c1}{\PYZsh{}\PYZsh{} Step 7 OUTPUT (\PYZsq{}The method failed after N\PYZhy{}0 iterations, N\PYZhy{}0 =\PYZsq{}, N\PYZus{}0) }
         \PY{c+c1}{\PYZsh{}\PYZsh{} (The procedure was unsuccessful. STOP)}
         \PY{k}{if} \PY{n}{i} \PY{o}{\PYZgt{}} \PY{n}{Iterations}\PY{p}{:}
             \PY{n+nb}{print}\PY{p}{(}\PY{n}{f}\PY{l+s+s1}{\PYZsq{}}\PY{l+s+s1}{The method failed after N\PYZus{}0 iterations, N\PYZus{}0 = }\PY{l+s+si}{\PYZob{}Iterations\PYZcb{}}\PY{l+s+s1}{\PYZsq{}}\PY{p}{)}
\end{Verbatim}


    \begin{Verbatim}[commandchars=\\\{\}]
The procedure was successful. p\_0: 1.4142135623746899 n: 4

    \end{Verbatim}

    \subsection{Compare Bisection Method Results To Newton's
Method}\label{compare-bisection-method-results-to-newtons-method}

    \begin{Verbatim}[commandchars=\\\{\}]
{\color{incolor}In [{\color{incolor}12}]:} \PY{c+c1}{\PYZsh{}\PYZsh{} Bisection method (pg. 49)}
         \PY{c+c1}{\PYZsh{}\PYZsh{} Testing for pg.50}
         \PY{c+c1}{\PYZsh{}\PYZsh{} Chapter 2.1 Example 1}
         
         \PY{c+c1}{\PYZsh{}\PYZsh{} INPUT endpoints a,b; tolerance TOL; maximum number of iterations N\PYZus{}0}
         
         \PY{c+c1}{\PYZsh{}\PYZsh{} Define the function}
         \PY{c+c1}{\PYZsh{}\PYZsh{} Maybe find an easy way to find the derivative?}
         \PY{k}{def} \PY{n+nf}{f}\PY{p}{(}\PY{n}{x}\PY{p}{)}\PY{p}{:}
             \PY{k}{return} \PY{n}{x}\PY{o}{*}\PY{o}{*}\PY{l+m+mi}{2} \PY{o}{\PYZhy{}} \PY{l+m+mi}{2}
         
         \PY{c+c1}{\PYZsh{}\PYZsh{} INPUT endpoints a,b; tolerance TOL; maximum number of iterations N\PYZus{}0}
         
         \PY{n}{a} \PY{o}{=} \PY{l+m+mi}{1}
         \PY{n}{b} \PY{o}{=} \PY{l+m+mi}{2}
         
         \PY{n}{TOL} \PY{o}{=} \PY{l+m+mi}{1} \PY{o}{*} \PY{l+m+mi}{10}\PY{o}{*}\PY{o}{*}\PY{o}{\PYZhy{}}\PY{l+m+mi}{6}
         \PY{n}{Iterations} \PY{o}{=} \PY{l+m+mi}{30}
         
         \PY{n}{results} \PY{o}{=} \PY{p}{[}\PY{p}{]}   \PY{c+c1}{\PYZsh{}\PYZsh{} Need to save the results to something}
         
         \PY{c+c1}{\PYZsh{}\PYZsh{} OUTPUT approximate solution p or message of failure}
         \PY{c+c1}{\PYZsh{}\PYZsh{} Step 1 Set i \PYZhy{} 1; FA = f(a)}
         
         \PY{n}{i} \PY{o}{=} \PY{l+m+mi}{1}
         \PY{n}{FA} \PY{o}{=} \PY{n}{f}\PY{p}{(}\PY{n}{a}\PY{p}{)}
         
         \PY{c+c1}{\PYZsh{}\PYZsh{} Step 2 While i \PYZlt{}= N\PYZus{}0 do steps 3\PYZhy{}6}
         
         \PY{k}{while} \PY{n}{i} \PY{o}{\PYZlt{}}\PY{o}{=} \PY{n}{Iterations}\PY{p}{:}
         
         \PY{c+c1}{\PYZsh{}\PYZsh{} Step 3 Set p = 1 + (b \PYZhy{} a)/2; (Compute P\PYZhy{}i)}
         
             \PY{n}{p} \PY{o}{=} \PY{n}{a} \PY{o}{+} \PY{p}{(}\PY{n}{b} \PY{o}{\PYZhy{}} \PY{n}{a}\PY{p}{)}\PY{o}{/}\PY{l+m+mi}{2}
             \PY{n}{FP} \PY{o}{=} \PY{n}{f}\PY{p}{(}\PY{n}{p}\PY{p}{)}
         
         \PY{c+c1}{\PYZsh{}\PYZsh{} Step 4 If FP = 0 or (b \PYZhy{} a)/2 \PYZlt{} TOL then OUTPUT(p); (Procedure completed successfully) STOP}
             \PY{k}{if} \PY{n}{FP} \PY{o}{==} \PY{l+m+mi}{0} \PY{o+ow}{or} \PY{p}{(}\PY{n}{b} \PY{o}{\PYZhy{}} \PY{n}{a}\PY{p}{)}\PY{o}{/}\PY{l+m+mi}{2} \PY{o}{\PYZlt{}} \PY{n}{TOL}\PY{p}{:}
                 \PY{n+nb}{print}\PY{p}{(}\PY{n}{f}\PY{l+s+s1}{\PYZsq{}}\PY{l+s+s1}{The procedure was successful. p: }\PY{l+s+si}{\PYZob{}p\PYZcb{}}\PY{l+s+s1}{ n: }\PY{l+s+si}{\PYZob{}i\PYZcb{}}\PY{l+s+s1}{\PYZsq{}}\PY{p}{)}
                 \PY{k}{break}
         \PY{c+c1}{\PYZsh{}\PYZsh{} Step 5 Set i = i + 1}
             \PY{n}{i} \PY{o}{=} \PY{n}{i} \PY{o}{+} \PY{l+m+mi}{1}
         \PY{c+c1}{\PYZsh{}\PYZsh{} Step 6 If FA * FP \PYZgt{} 0 then set a = p; (Compute a\PYZus{}i, b\PYZus{}i) FA = FP; else set b = p. (FA is unchanged)}
             \PY{k}{if} \PY{n}{FA} \PY{o}{*} \PY{n}{FP} \PY{o}{\PYZgt{}} \PY{l+m+mi}{0}\PY{p}{:}
                 \PY{n}{a} \PY{o}{=} \PY{n}{p}
                 \PY{n}{FA} \PY{o}{=} \PY{n}{FP}
             \PY{k}{else}\PY{p}{:}
                 \PY{n}{b} \PY{o}{=} \PY{n}{p}
                 
             \PY{n}{results}\PY{o}{.}\PY{n}{append}\PY{p}{(}\PY{n}{p}\PY{p}{)}
         \PY{c+c1}{\PYZsh{}\PYZsh{} Step 7 OUTPUT (\PYZsq{}Method failed after N\PYZus{}0 iterations, N\PYZus{}0 =\PYZsq{}, N\PYZus{}0); STOP}
         \PY{k}{if} \PY{n}{i} \PY{o}{\PYZgt{}} \PY{n}{Iterations}\PY{p}{:}
             \PY{n+nb}{print}\PY{p}{(}\PY{n}{f}\PY{l+s+s1}{\PYZsq{}}\PY{l+s+s1}{The method failed after N\PYZus{}0 iterations, N\PYZus{}0 = }\PY{l+s+si}{\PYZob{}Iterations\PYZcb{}}\PY{l+s+s1}{\PYZsq{}}\PY{p}{)}
\end{Verbatim}


    \begin{Verbatim}[commandchars=\\\{\}]
The procedure was successful. p: 1.4142141342163086 n: 20

    \end{Verbatim}

    \subsection{Compare Secant Method Results To Newton's
Method}\label{compare-secant-method-results-to-newtons-method}

    \begin{Verbatim}[commandchars=\\\{\}]
{\color{incolor}In [{\color{incolor}13}]:} \PY{c+c1}{\PYZsh{}\PYZsh{} Secant method}
         \PY{c+c1}{\PYZsh{}\PYZsh{} Testing for pg.71}
         \PY{c+c1}{\PYZsh{}\PYZsh{} Chapter 2.3 Example 2}
         
         \PY{c+c1}{\PYZsh{} Define the function}
         
         \PY{k}{def} \PY{n+nf}{f}\PY{p}{(}\PY{n}{x}\PY{p}{)}\PY{p}{:}
             \PY{k}{return} \PY{n}{x}\PY{o}{*}\PY{o}{*}\PY{l+m+mi}{2} \PY{o}{\PYZhy{}} \PY{l+m+mi}{2}
         
         \PY{c+c1}{\PYZsh{}\PYZsh{} Define bounderies}
         
         \PY{n}{a} \PY{o}{=} \PY{l+m+mi}{1}
         \PY{n}{b} \PY{o}{=} \PY{l+m+mi}{2}
         
         \PY{c+c1}{\PYZsh{}\PYZsh{} INPUT initial approximations p\PYZus{}0, p; tolerance TOL; maximum number of iterations N\PYZus{}0.}
         
         \PY{n}{p\PYZus{}0} \PY{o}{=} \PY{l+m+mi}{1}
         \PY{n}{p\PYZus{}1} \PY{o}{=} \PY{l+m+mi}{2}
         \PY{n}{TOL} \PY{o}{=} \PY{l+m+mi}{1} \PY{o}{*} \PY{l+m+mi}{10}\PY{o}{*}\PY{o}{*}\PY{o}{\PYZhy{}}\PY{l+m+mi}{6}
         \PY{n}{Iterations} \PY{o}{=} \PY{l+m+mi}{20}
         
         \PY{n}{results} \PY{o}{=} \PY{p}{[}\PY{p}{]}   \PY{c+c1}{\PYZsh{}\PYZsh{} Need to save the results to something}
         
         \PY{c+c1}{\PYZsh{}\PYZsh{} Step 1 set i = 2; q\PYZus{}0 = f(p\PYZus{}0); q\PYZus{}1 = f(p\PYZus{}1)}
         
         \PY{n}{i} \PY{o}{=} \PY{l+m+mi}{2}
         \PY{n}{q\PYZus{}0} \PY{o}{=} \PY{n}{f}\PY{p}{(}\PY{n}{p\PYZus{}0}\PY{p}{)}
         \PY{n}{q\PYZus{}1} \PY{o}{=} \PY{n}{f}\PY{p}{(}\PY{n}{p\PYZus{}1}\PY{p}{)}
         
         \PY{c+c1}{\PYZsh{}\PYZsh{} Step 2 While i \PYZlt{}= N\PYZus{}0 do Steps 3\PYZhy{}6}
         
         \PY{k}{while} \PY{n}{i} \PY{o}{\PYZlt{}}\PY{o}{=} \PY{n}{Iterations}\PY{p}{:}
             
         \PY{c+c1}{\PYZsh{}\PYZsh{} Step 3 Set p = p\PYZus{}1 \PYZhy{} q\PYZus{}1(p\PYZus{}1 \PYZhy{} p\PYZus{}0)/(q\PYZus{}1 \PYZhy{} q\PYZus{}0)    (Compute p\PYZus{}i)}
             \PY{n}{p} \PY{o}{=} \PY{n}{p\PYZus{}1} \PY{o}{\PYZhy{}} \PY{p}{(}\PY{n}{q\PYZus{}1} \PY{o}{*} \PY{p}{(}\PY{n}{p\PYZus{}1} \PY{o}{\PYZhy{}} \PY{n}{p\PYZus{}0}\PY{p}{)}\PY{o}{/}\PY{p}{(}\PY{n}{q\PYZus{}1} \PY{o}{\PYZhy{}} \PY{n}{q\PYZus{}0}\PY{p}{)}\PY{p}{)}
         \PY{c+c1}{\PYZsh{}\PYZsh{} Step 4 If |p \PYZhy{} p\PYZus{}1| \PYZlt{} TOL then OUTPUT(p); (The procedure was successful.) STOP}
             \PY{k}{if} \PY{n}{np}\PY{o}{.}\PY{n}{absolute}\PY{p}{(}\PY{n}{p} \PY{o}{\PYZhy{}} \PY{n}{p\PYZus{}1}\PY{p}{)} \PY{o}{\PYZlt{}} \PY{n}{TOL}\PY{p}{:}
                 \PY{n+nb}{print}\PY{p}{(}\PY{n}{f}\PY{l+s+s1}{\PYZsq{}}\PY{l+s+s1}{The procedure was successful. p: }\PY{l+s+si}{\PYZob{}p\PYZcb{}}\PY{l+s+s1}{ n: }\PY{l+s+s1}{\PYZob{}}\PY{l+s+s1}{i \PYZhy{} 2\PYZcb{}}\PY{l+s+s1}{\PYZsq{}}\PY{p}{)}
                 \PY{k}{break}
         \PY{c+c1}{\PYZsh{}\PYZsh{} Step 5 Set i = i + 1}
             \PY{n}{i} \PY{o}{=} \PY{n}{i} \PY{o}{+} \PY{l+m+mi}{1}
         \PY{c+c1}{\PYZsh{}\PYZsh{} Step 6 Set p\PYZus{}0 = p\PYZus{}1; q\PYZus{}0 = q\PYZus{}1; p\PYZus{}1 = p; q\PYZus{}1 = f(p) (Update P\PYZus{}0, q\PYZus{}0, p\PYZus{}1, q\PYZus{}1)}
             \PY{n}{p\PYZus{}0} \PY{o}{=} \PY{n}{p\PYZus{}1}
             \PY{n}{q\PYZus{}0} \PY{o}{=} \PY{n}{q\PYZus{}1}
             \PY{n}{p\PYZus{}1} \PY{o}{=} \PY{n}{p}
             \PY{n}{q\PYZus{}1} \PY{o}{=} \PY{n}{f}\PY{p}{(}\PY{n}{p}\PY{p}{)}
             
             \PY{n}{results}\PY{o}{.}\PY{n}{append}\PY{p}{(}\PY{n}{p}\PY{p}{)}
         \PY{c+c1}{\PYZsh{}\PYZsh{} Step 7 OUTPUT (\PYZsq{}The method failed after N\PYZus{}0 iterations, N\PYZus{}0 =\PYZsq{}, N\PYZus{}0);}
             \PY{c+c1}{\PYZsh{}\PYZsh{} (The procedure was unsuccessful.) STOP}
         \PY{k}{if} \PY{n}{i} \PY{o}{\PYZgt{}} \PY{n}{Iterations}\PY{p}{:}
             \PY{n+nb}{print}\PY{p}{(}\PY{n}{f}\PY{l+s+s1}{\PYZsq{}}\PY{l+s+s1}{The method failed after N\PYZus{}0 iterations, N\PYZus{}0 = }\PY{l+s+si}{\PYZob{}Iterations\PYZcb{}}\PY{l+s+s1}{\PYZsq{}}\PY{p}{)}
\end{Verbatim}


    \begin{Verbatim}[commandchars=\\\{\}]
The procedure was successful. p: 1.4142135623730954 n: 5

    \end{Verbatim}


    % Add a bibliography block to the postdoc
    
    
    
    \end{document}
